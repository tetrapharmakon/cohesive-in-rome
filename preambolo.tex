\usepackage[utf8]{inputenc}
\usepackage[T1]{fontenc}
\usepackage{graphicx}
\usepackage{grffile}
\usepackage{longtable}
\usepackage{wrapfig,epigraph}
\usepackage{rotating}
\usepackage[normalem]{ulem}
\usepackage{amsmath}
\usepackage{textcomp}
\usepackage{amssymb}
\usepackage{capt-of}
\usepackage{hyperref, tipa}
\let\pb\relax

\usepackage{tikz}
\usetikzlibrary{calc}
\makeatletter
\def\th@mystyle{%
    \normalfont % body font
    \setbeamercolor{block title example}{bg=red!5,fg=red!65!black}
    \setbeamercolor{block body example}{ bg=red!5,fg=black}
    \def\inserttheoremblockenv{exampleblock}
  }
\makeatother

\newcommand{\xto}[1]{\xrightarrow{#1}}
\newcommand{\xot}[1]{\xleftarrow{#1}}

\theoremstyle{mystyle}
	\newtheorem{df}{Definition}
	\newtheorem{oss}{Remark}
	\newtheorem{cor}{Corollary}
	\newtheorem{thm}{Theorem}
	\newtheorem{prop}{Proposition}
	\newtheorem{es}{Example}

\def\yo{y}
\def\shape{\text{\textesh}}
\def\thH{\widetilde{\clH}}
\def\disc{\textsc{disc}}
\def\codisc{\textsc{codisc}}

\usepackage{xparse}
\usepackage[color,all,2cell]{xy}\UseAllTwocells

\colorlet{dagreen}{green!70!black}

\def\quadruple#1{%
	\ar@<12pt>@{^{(}->}[r]%
	\ar@{<-}@<4pt>[r]|{i^*_{#1}}%
	\ar@<-4pt>@{^{(}->}[r]|{i_{*,#1}}%
	\ar@<-12pt>@{<-}[r]%
 }

\ExplSyntaxOn
\NewDocumentCommand{\makeabbrev}{mmm}
 {
  \yoruk_makeabbrev:nnn { #1 } { #2 } { #3 }
 }

\cs_new_protected:Npn \yoruk_makeabbrev:nnn #1 #2 #3
 {
  \clist_map_inline:nn { #3 }
   {
    \cs_new_protected:cpn { #2 } { #1 { ##1 } }
   }
 }
\ExplSyntaxOff

\makeabbrev{\textbf}{bf#1}{
  a,b,c,d,e,g,h,i,j,k,l,m,n,o,p,q,r,t,u,v,w,x,y,z,%
  A,B,C,D,E,G,H,I,J,K,L,M,N,O,P,Q,R,T,U,V,W,X,Y,Z }
\makeabbrev{\boldsymbol}{bs#1}{%
    a,b,c,d,e,f,g,h,i,j,k,l,m,n,o,p,q,r,s,t,u,v,w,x,y,z,%
    A,B,C,D,E,F,G,H,I,J,K,L,M,N,O,P,Q,R,S,T,U,V,W,X,Y,Z }
\makeabbrev{\mathsf}{sf#1}{
  a,b,c,d,e,f,g,h,i,j,k,l,m,n,o,p,q,r,s,t,u,v,w,x,y,z,%
  A,B,C,D,E,F,G,H,I,J,K,L,M,N,O,P,Q,R,S,T,U,V,W,X,Y,Z }
\makeabbrev{\mathfrak}{fk#1}{
  a,b,c,d,e,f,g,h,j,k,i,l,m,n,o,p,q,r,s,t,u,v,w,x,y,z,%
  A,B,C,D,E,F,G,H,I,J,K,L,M,N,O,P,Q,R,S,T,U,V,W,X,Y,Z }
\makeabbrev{\mathcal}{cl#1}{
  A,B,C,D,E,F,G,H,I,J,K,L,M,N,O,P,Q,R,S,T,U,V,W,X,Y,Z }
\makeabbrev{\mathbb}{bb#1}{
  A,B,C,D,E,F,G,H,I,J,K,L,M,N,O,P,Q,R,S,T,U,V,W,X,Y,Z }

\newlength{\seplen}
\setlength{\seplen}{5pt}
\newlength{\sepwid}
\setlength{\sepwid}{.4pt}
\def\firstblank{\,\rule{\seplen}{\sepwid}\,}
\def\secondblank{\firstblank\llap{\raisebox{2pt}{\firstblank}}}
\def\To{\Rightarrow}

\def\Set{\underline{\text{Set}}}
\newcommand{\op}{\text{op}}

\setbeamertemplate{navigation symbols}{}
\usepackage{animate}

\newenvironment{variableblock}[3]{%
  \setbeamercolor{block body}{#2}
  \setbeamercolor{block title}{#3}
  \begin{block}{#1}}{\end{block}}

\newenvironment{myblock}[1]{%
	\begin{variableblock}{#1}{bg=blue!10, fg=black}{bg=blue!10, fg=blue}%
	}{%
	\end{variableblock}%
	}


\usetheme{metropolis}